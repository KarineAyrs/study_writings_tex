%!TEX root = ../slides2.tex

\section{Referring object segmentation}

\begin{frame}
  \frametitle{Referring object segmentation}
  \begin{itemize}
    \item \textbf{Referring video object segmentation (R-VOS)} -задача сегментации объекта на видео по его описанию на натуральном языке.
    \item \textbf{Referring image segmentation} - задача сегментации объекта на изображении по его описанию на натуральном языке.
    
  \end{itemize}
\end{frame}


\section{Актуальность}

\begin{frame}
  \frametitle{Актуальность}
  \begin{itemize}
    \item видеонаблюдение 
    \item приложения для редактирования изображений и видео 
    \item обнаружение патологий на медицинских изображениях
    \item взаимодействие человека и робота посредством языка
  \end{itemize}
  
\end{frame}


\section{Цель}
\begin{frame}
    \frametitle{Цель}
    \begin{itemize}
        \item Дан видеоклип \(V=\{I_t\}_{t=1}^{T}\), содержащий \(T\) фреймов и 
        текстовое описание \(E=\{e_l\}_{l=1}^L\), состоящее из \(L\) слов. \\
        Цель: создать маску сегментации описываемого объекта 
        \(S=\{s_t\}_{t=1}^T, s_t \in \mathbb{R}^{HxW}\), для каждого фрейма. 
        \item Дано изображение \(I\) и текстовое описание \(E\). \\Цель: создать маску сегментации описываемого объекта.

    \end{itemize}
\end{frame}
