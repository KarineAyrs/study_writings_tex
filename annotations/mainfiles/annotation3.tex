\section{On the Compactness, Efficiency, and Representation of 3D Convolutional Networks: Brain Parcellation as a Pretext Task}

\subsection*{Ссылка}\url{https://arxiv.org/abs/1707.01992}
\subsection*{Введение}
На сегодняшний день большинство исследований в области обработки медицинских 
изображений нейронными сетями проводятся с использованием двумерных изображений, в то время как 
трехмерные изображения являются более информативными. Однако, анализ и обработка трехмерных изображений сопряжены 
с вычислительными трудностями и, в то время как разработка эффективной и рабочей нейронной сети 
трехмерной архитектуры представляет большой интерес, ее проектирование остается сложной задачей.
 Целью данной работы является разработка компактной сетевой архитектуры высокого разрешения для сегментации 
 структур объемных изображений. Также демонстрируется возможность оценки неопределенности на воксельном уровне 
 с помощью метода Монте-Карло на предлагаемой сети во время тестирования.
\subsection*{Основная идея}
Нейронная сеть, предложенная в данной статье состоит из 20 сверточных слоев.
 В первых семи слоях ядро свертки имеет размерность 3x3x3, они необходимы для 
 локализации низкоуровневых объектов изображения - краев и углов. В последующих сверточных 
 слоях размерность ядра увеличивается в два или четыре раза - они необходимы для локализации 
 более значительных фрагментов. Каждые два сверточных слоя группируются в residual block. 
 Внутри каждого такого блока каждый сверточный слой связан поэлементно с ReLU и со слоем batch нормализации. 
 Сеть обучается от начала и до конца, входные данные предобрабатываются (стандартизация и аугментация). 
\subsection*{Данные}
Для демонстрации возможности обучения на сложных трехмерных изображениях были выбраны 543 МРТ - 
изображения, снятых в режиме Т1-ВИ из датасета ADNI. 
\subsection*{Результаты}
Используя сеть предложенной архитектуры (HC-default) авторы сравнивают
качество ее предсказания с предсказаниями, полученными от трех вариаций данной сети: 
(1) HC-default без residual blocks и с логарифмической функцией потерь (NoRes-entropy); (2) HC-default без residual blocks с
 коэффициентом Серенсена в качестве функции потерь (NoRes-dice); (3) HC-default с дополнительным dropout слоем (HC-dropout). 
 К тому же, был проведен сравнительный анализ с тремя уже существующими сетями для трехмерной сегментации - 3D U-net, V-net, 
 Deepmedic. Было установлено, что тренировка сетей с логарифмической функцией потерь ведет к низким результатам сегментации.
  Поэтому, в качестве функции потерь был выбран коэффициент Серенсена (Dice-coefficient). С относительно маленьким количеством параметров, 
  HC-default и HC-dropout превосходят вышеперечисленные модели по метрике Серенсена. Это означает, что предложенная сеть лучше справляется с
   поставленной задачей. 
\subsection*{Заключение}
В данной работе была продемонстрирована архитектура трехмерной сверточной сети, 
которая включает в себя слои свертки и residual block'и. Данная сеть концептуально 
проще и имеет меньшее количество параметров, чем уже существующие сети для обработки 
трехмерных изображений. Более того, по сравнению с ними она показала лучшие результаты 
в задачах сегментации и парцелляции головного мозга. 