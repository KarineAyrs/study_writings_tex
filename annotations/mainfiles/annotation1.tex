\section{Deep reinforcement learning-based image classification achieves perfect testing set accuracy for MRI brain tumors with a training set of only 30 images}

\subsection*{Ссылка}\url{https://arxiv.org/abs/2102.02895}

\subsection*{Введение}
Задачи классификации и сегментации являются основной областью применения искусственного 
интеллекта в радиологии и попадают в категорию задач, решаемых с помощью метода глубокого обучения с учителем. 
Однако, применение данного метода в медицине имеет свои ограничения: для реализации требуется большое количество размеченных данных 
квалифицированными специалистами; обобщающая способность падает, когда требуется сделать предсказание на изображениях со сканеров, отличных от тех, 
на которых обучалась сеть, либо на изображениях с других медицинских учреждений. Немаловажен и феномен \glqq черного ящика\grqq, при котором 
не до конца понятно, как получены результаты и доверие к методу среди специалистов и пациентов падает. 
\subsection*{Основная идея}
В своих предыдущих работах авторы статьи предложили метод обучения с подкреплением в радиологии и показали, 
что с его помощью решаются задачи локализации и сегментации пораженной области на изображении. В данной работе для отыскания 
оптимальной стратегии авторы использовали Марковский процесс принятия решений. Таким образом, черно-белое изображение 
перекрывается красным, отображая начальное состояние. Далее, на каждом шаге агент совершает действие - предсказание, 
результатом которого является 0 - нормальное изображение, и 1 - изображение, содержащее опухоль. Если предсказан верный класс, 
то в следующем состоянии изображение преобразуется в черно-белое с зеленым перекрытием. В противном случае, изображение остается 
красным либо с зеленого меняется на красный. За правильное предсказание агент награждается в размере +1, а за неверное штрафуется 
в размере -1. Основная цель обучения с подкреплением - достичь максимальной суммарной награды. Тренировка основывается 
на сочетании глубокой Q-сети (DQN) с TD(0) Q-обучением. Также, для сравнения классификации, основанной на глубоком обучении 
с учителем и обучении с подкреплением, авторы обучили сверточную нейронную сеть с архитектурой, схожей с архитектурой DQN на таком же наборе 
тренировочных данных.
\subsection*{Данные}
В качестве данных для обучения были выбраны 60 двумерных срезов трехмерных изображений из датасета BraTS 2020 Challenge tumor database. 
Все изображения были сняты в режиме T1-ВИ после введения контрастного вещества. 30 из них были размечены специалистами как нормальные,
а оставшиеся 30 - содержат злокачественные глиомы. Далее, 30 изображений из 60 были выбраны в качестве обучающего множества и 
30 в качестве тренировочного (в каждом по 15 нормальных и злокачественных изображений).
\subsection*{Результаты}
Рассматривая точность на обучающем множестве в зависимости от времени обучения с подкреплением можно видеть постепенное 
повышение обобщающей способности, а точность в 100\%  достигается через 200 эпизодов обучения. В то же время, 
сверточная нейронная сеть быстро переобучается на таком маленьком наборе данных и точность предсказания достигает лишь 57\%.
\subsection*{Заключение}
Учитывая все вышеизложенное и тот факт, что зачастую медицинские наборы данных очень малы, а в данном исследовании 
\glqqтрадиционная\grqq\quad нейронная сеть быстро переобучается на маленьком датасете, авторы показали, 
что обучение с подкреплением показывает значительное преимущество в задаче классификации, сегментации 
и локализации (эти факты показаны в предыдущих исследованиях). Однако, использование двумерного среза 
вместо целого трехмерного изображения является ограничением, ровно как и то, что предсказывалось только два класса.
