\section{Localization of Critical Findings in Chest X-Ray without Local Annotations Using Multi-Instance Learning}

\subsection*{Ссылка} \url{https://arxiv.org/abs/2001.08817}
\subsection*{Введение}
Автоматическое детектирование серьезных нарушений легких, таких как пневмоторакс, пневмония и отек по рентгеновским 
изображениям широко исследуется на данный момент. Для решения задачи классификации изображений обычно используются 
сверточные нейронные сети, однако остается открытым вопрос интерпретации и объяснения результата предсказания и в 
медицине он особенно важен в разрезе доверия данному методу. Существуют методы, которые строят тепловую карту изображения 
в пикселях, указывая регионы, которые участвовали в прогнозировании. Однако, они применяются уже после классификации и 
тепловые карты генерируются фильтрами с низким разрешением, а после проецируются обратно до размеров исходного изображения, 
что может плохо сказаться на локализации в случае медицинских изображений, которые обычно имеют высокое разрешение. 
Также, широко используются алгоритмы обнаружения объектов и сегментации, выделяющие границу региона, однако они требуют 
попиксельную маркировку (маску), на основе которой будет строиться предсказание. Здесь и находится ограничение - 
разметка медицинских изображений требует наличие квалифицированных специалистов и является трудоемкой и времязатратной задачей.
\subsection*{Основная идея}
В данной работе авторы ставят задачей устранить два вышеперечисленных недостатка - 
низкую интерпретируемость и необходимость локальной аннотации на основе технологии многовариантного 
обучения (multi-instance learning (MIL)). При таком подходе данные разделяют на множество частей, 
называемых \textit{вариантами}, которые совместно анализируются для понимания того, какой вклад они 
внесли в предсказание метки класса. В данном случае в качестве вариантов выступают части изображения. 
Цель данного подхода в бинарной классификации рентгеновских изображений - предсказать метку для каждой части (0 или 1), 
что является обучением со слабой разметкой, так как известна метка для целого изображения, но не для его части.
 Очевидно, что в изображениях, не содержащих аномалий, все части также не будут их содержать, а в изображениях с 
 патологическими нарушениями, хотя бы одна часть должна быть помечена как дефектная. Таким образом, для классификации, 
 части подаются на вход сверточной нейронной сети, которая на выходе выдает вероятность содержания аномалии от 0 до 1 и, 
 так как части не имеют разметки, то в процессе обучения MIL использует механизм, выявляющий связь между меткой, 
 присущей всему изображению и предсказываемой меткой. Полученные значения меток в негативных частях (тех, которые не содержат аномалий), 
 подавляются до 0, а в позитивных - вытягиваются к 1, таким образом происходит непрерывная классификация позитивных и негативных 
 изображений и определяются части, ответственные за принятие сетью решения.
\subsection*{Данные}
Авторы использовали три датасета - UWMC (пневмоторакс), RSNA/Kaggle (пневмония), MIMIC-CXR (отек легких).\par 

\subsection*{Результаты}
В результате экспериментов, используя сеть VGG16, авторы достигли точности в 0.89, 0.84 и 0.82 для 
пневмоторакса, пневмонии и отека легких соответственно. Также, в качестве дополнительного эксперимента
 по классификации пневмоторакса из датасета UWMC было произведено сравнение метода MIL 
 с двумя другими методами классификации на основе модифицированной сети ResNet50 и 
 полносвязной сети (FCN). Получены следующие результаты: 0.96 (ResNet50), 0.93 (MIL), 0.92 (FCN). 
 Также, в статье авторы приводят визуализацию результата работы метода MIL с соответствующей интерпретацией - 
 части, которые содержат патологии с вероятностью, близкой к 1 толстые и обведены темно-красным, части со средним показателем - 
 светло-красные и светлее с меткой близкой к нулю.
\subsection*{Заключение}
В данной статье описан и применен метод многовариантного обучения MIL, который одновременно 
классифицирует изображения и позволяет локализовать патологии без специальной разметки, 
имея только разделение на классы целых изображений, что позволяет понимать, какая 
часть изображения внесла больший вклад в результат работы сети.  Авторы утверждают, что данный метод масштабируем - 
его можно использовать для нахождения любого числа патологий на изображении.