\subsection*{Deep Kernel Representation for Image Reconstruction in PET}

% \subsection*{Ссылка} \url{https://arxiv.org/abs/2110.01174}
\subsubsection*{Введение} 
Реконструкция ПЭТ изображения является сложной задачей из-за низкого разрешения и высокого шума в данных. 
Среди разных методов реконструкции ПЭТ изображений, ядровые методы (kernel methods) шают проблему шума 
путем интеграции в изображение дополнительной информации. Дополнительную информацию можно получить из 
составных изображений динамического ПЭТ сканирования или из анатомических изображений (например, МРТ при совместном исследовании ПЭТ/МРТ). 
В существующих ядерных методах ядро обычно строится при помощи эмпирического подбора векторов признаков и ручного выбора параметров, связанных с методом. \par
\subsubsection*{Основная идея}
В данной работе \cite{ann5} описывается эквивалентность между представлением ядра в ядерном методе и 
обучаемой нейросетевой моделью. Основываясь на этой связи, далее предлагается метод \glqq глубокого ядра\grqq, 
который изучает обученные компоненты нейросетевой модели на доступных снимках, чтобы достичь автоматизации обучения, 
основанной на данных для оптимизированной ядерной модели. Далее, обученная ядерная модель применяется для реконструкции 
ПЭТ изображений и ожидается, что данный метод будет превосходить другие ядерные методы, основанные на эмпирических заключениях. 
Описываемый метод имеет уникальное преимущество - после обучения модели неизвестный ядерные коэффициенты остаются линейными и 
легко восстанавливаются по ПЭТ данным. К тому же, для этого не требуется большой набор данных. \par
\subsubsection*{Данные}
В качестве данных были использованы снимки динамического ПЭТ сканирования с помощью сканера GE 690, 
данные пациентов из UC Davis Medical Center со сканера GE Discovery ST PET/CT. \par
\subsubsection*{Результаты}
В работе наряду с предложенным методом приводятся существующие методы восстановления 
изображений, а далее с их помощью моделируются данные. Смоделированные данные были восстановлены с 
помощью четырех различных методов: (1) стандартная ML-EM реконструкция; (2) существующий ядерный метод без обучения; 
(3) предлагаемый метод глубокого ядра с онлайн обучением для извлечения признаков; (4) метод реконструкции DIP. 
Так, к примеру изображения, восстановленные с помощью ML-EM метода получились очень шумными, DIP метод привел к сильному сглаживанию, 
а восстановленные изображения с помощью описанного метода показали более четкие контуры и более низкий шум в левом и правом желудочке и миокарде. \par
\subsubsection*{Заключение}
Таким образом, авторы разработали новый ядерный метод для реконструкции ПЭТ изображений, который показывает более 
оптимальное обучение ядра, чем в эмпирических методах. Результаты компьютерного моделирования и реального набора 
данных показывают, что предложенный метод превосходит существующие ядерные и нейросетевые методы реконструкции ПЭТ изображений.