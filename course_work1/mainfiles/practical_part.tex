\section{Результаты}
Был проведен обзор двадцати статей, включающих в себя методы по сегментации злокачественных образований как в головном мозге, так и в других 
жизненно важных органах человека. Методы применялись к данным различных модальностей: МРТ, ПЭТ, КТ как по отдельности, так и в совокупности. Также, 
были представлены методы для реконструкции/восстановления ПЭТ-изображений по данным других исследований (МРТ, КТ). Полная таблица рассмотренных статей представлена ниже: 


{\small 

\begin{flushleft}
    

\captionof{table}{Общая таблица рассмотренных методов}
\resizebox{\columnwidth}{!}{
\begin{tabular}{|c|c|}

    \hline
    Модальность  & Метод \\ 

    \hline 
    \multirow{4}{*}{МРТ} & DQN+TD(0)\cite{ann1}, Brain Parcellation\cite{ann3}, Cascaded Unet\cite{Cascaded} \\ 
                         \cline{2-2}
                         &3D Self-Supervised Methods for Medical Imaging \cite{3DSelfSuper} \\
                         \cline{2-2}
                         & 3DSimCLR\cite{ann14}, Medical Transformer\cite{MedT}, U-net Transfromer\cite{ann19}, \\
                         \cline{2-2}
                         &  AFTer-Unet\cite{ATerUnet}, PGL \cite{ann15}, CaraNet\cite{CaraNet}, ViT-V-Net \cite{VitVNet}\\

    \hline 
    Рентген & MIL\cite{ann2}, MEDUSA\cite{ann17} \\
    
    \hline 
    ПЭТ/КТ & ES-Unet\cite{ann4}, NormResSE-Unet3+\cite{NormRes}, PET images from CT\cite{ann10}   \\
      
    \hline
    ПЭТ & Deep Kernel Representation\cite{ann5}, Direct PET Image Reconstruction\cite{ann9}\\
                      

    \hline 
    МРТ/КТ & Effects of Self-Attention for Medical Image Analysis \cite{ann16} \\
    \hline
    Другое & Is it Time to Replace CNNs with Transformers for Medical Images? \cite{ann11} \\
        
        \hline
        \hline
\end{tabular}
}
\end{flushleft}
}