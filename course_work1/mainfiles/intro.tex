%!TEX root = ../course_work.tex
% \phantomsection
\section{Введение} 
% \addcontentsline{toc}{section}{Введение}



Термин «первичные опухоли ЦНС» (ПО ЦНС) объединяет различные по
гистологическому строению, злокачественности и клиническому течению опухоли,
общим для которых является происхождение из тканей, составляющих центральную
нервную систему и ее оболочки. \cite{MedicTherms} \par


Опухоли головного мозга (ОГМ)- это группа различных внутричерепных новообразований 
(доброкачественных и злокачественных), возникающих вследствие запуска процесса 
аномального неконтролируемого деления клеток, которые в прошлом являлись 
нормальными составляющими различных тканей головного мозга, или возникающих 
вследствие метастазирования первичной опухоли, находящейся в другом органе. 
На возникновение новообразований влияют различные факторы: некоторые внутренние 
редкие дефекты генов, внешние воздействия - ультрафиолетовые лучи или 
рентгеновское излучение, химические вещества, инфекции, - приводят 
к спонтанному появлению мутации, что повышает риск развития 
онкологических заболеваний. Опухоли головного мозга с разными частотами затрагивают все возрастные категории 
населения. Прогноз заболевания в большинстве случаев остается неблагоприятным. Продолжительность 
жизни больных с ОГМ значительно варьируется в зависимости от типа новообразования, составляя 
в среднем от 1 года до 3-7 лет. Подобные новообразования встречаются с частотой от 5 до 7,5 
случаев на 100 тысяч начеления. Поэтому одним из наиболее значимых приоритетных направлений 
современной медицины является совершенствование существующих методов диагностики и терапии ОГМ. 
\cite{MolNeuro}


Одной из наиболее часто встречающихся 
злокачественных патологий центральной нервной
системы является глиальная опухоль. \par 

Глиомы – это собирательный термин, который объединяет все диффузные
астроцитарные и олигодендроглиальные опухоли, а также другие виды – пилоидную
астроцитому, субэпендимарную гигантоклеточную астроцитому, астробластому и
другие опухоли, исходящие из клеток глии. \cite{MedicTherms}

Опухоли ЦНС очень разнообразны. Их классифицируют по локализации, гистологическому типу, степени злокачественности.

\begin{minipage}{1.0\linewidth}
    \begin{center}
    
    \includegraphics[scale=0.07]{Astrocytoma.jpg} 
    % \caption{\scriptsize{Пайплайн для задачи предсказания выживаемости. Состоит из трех шагов:
    % сбор клинических данных, изображений и препроцессинга. Затем, выбираются извлеченные признаки и производится 
    % предсказание выживаемости.}}
\end{center}
\end{minipage}

Классификация глиом \cite{MedicTherms}:
\begin{enumerate}
    \item Астроцитома – опухоль, развивающаяся из астроцитарной части глии и
    представленная астроцитами. Может локализоваться как в больших полушариях мозга, так
    и в мозжечке, а также в стволе головного мозга и спинном мозге. Различают астроцитомы
    низкой и высокой степени злокачественности.
    \item Олигодендроглиома и олигоастроцитома – опухоли, преимущественно состоящие
    из олигодендроцитов.
    \item Глиоматоз головного мозга – это диффузное поражение глиомой структур головного
    мозга (более 3-х анатомических областей больших полушарий, обычно с переходом через
    мозолистое тело и с перивентрикулярным распространением). 
    \item Глиомы ствола головного мозга. На разных уровнях поражения ствола головного мозга
    встречаются различные глиальные опухоли. Часть этих опухолей носит доброкачественный характер и может не
    прогрессировать без лечения в течение всей жизни человека. Другие характеризуются, напротив, агрессивным течением с
    ограниченными возможностями специализированной помощи этим больным. 
   \item Глиомы спинного мозга. Как правило, диффузные интрамедуллярные опухоли,
    поражающие различные уровни спинного мозга. 
    \item и т.д.
\end{enumerate} 





Методы обследования глиальных опухолей: 
\begin{enumerate}
    \item Компьютерная томография с контрастным усилением (КТ) - это специальный метод, в котором 
    используется рентгеновское излучение. С его помощью тело человека послойно просвечивают рентгеновскими лучами. 

    \item  Магнитно-резонансная томография  с контрастным усилением(МРТ) - 
    метод визуализации, основанный на резонансе атомов водорода в организме 
    человека на магнитное поле,  создаваемое томографом.


    \item  Позитронно-эмиссионная томография  (ПЭТ) - технология визуализации, основанная на количественной и качественной оценке биохимических процессов, происходящих в тканях in vivo. 
    \item  Сцинтиграфия  - используется для оценки функционирования различных органов и тканей. Такие методы диагностики, как рентген, УЗИ, КТ или МРТ ориентированы на выявление структурных изменений в тканях организма, и не всегда способны различить болезнь на ранних её стадиях, когда отклонения проявились на уровне биохимических изменений в тканях. В это время приходит на помощь сцинтиграфия, которую поэтому и называют молекулярной диагностикой
    \item Неврологическое исследование, которое обязательно включает в себя офтальмологическую проверку остроты зрения, глазного дна и полей зрения
    \item Ангиография -  класс методов контрастного исследования кровеносных сосудов.

\end{enumerate}


На данный момент «золотым стандартом» в диагностике объемных образований головного мозга, 
определении степени злокачественности, тактики лечения и прогноза заболевания 
является магнитно-резонансная томография (МРТ) с контрастным усилением. 
Однако методы диагностики, направленные, главным образом, на оценку 
структурных изменений мозга, к которым относится и МРТ, обладают 
невысокой специфичностью в выявлении микроструктурных и  метаболических 
перестроек в опухолевой ткани, что ограничивает раннюю диагностику глиального 
образования. Накопление МР-контрастного препарата не всегда напрямую коррелирует 
со степенью злокачественности опухоли, поэтому всё чаще и чаще для диагностики применяется ПЭТ-исследование 
совместно с КТ и МРТ исследованиями для уточнения результатов. \par

Ранняя диагностика опухоли способствует высоким результатам лечения.
Заболевание нуждается в дифференцировке от гематомы внутри мозга, абсцесса,
эпилепсии, прочих опухолевых процессов в центральной нервной системе,
последствий инсульта. \par 

В настоящее время все больше внимания уделяется автоматическим методам диагностики злокачественных 
новообразований (классификация, сегментация), разработанных на основе нейронных сетей, что в перспективе 
позволит ускорить процесс диагностики патологического процесса, сокращая время между проведением исследования и началом лечения, а также уменьшить
нагрузку на врачей и медработников, которые смогут более эффективно проводить лечение заболевания.

В данной работе проводится обзор различных нейросетевых методов, которые решают задачи сегментации, классификации и 
реконструкции медицинских изображений различных моадльностей (ПЭТ, МРТ, КТ), которые могут 
применяться для автоматической диагностики злокачественных заболеваний.
Сложность работы с медицинскими данными заключается в том, что зачастую качественных наборов данных очень мало, а для разметки нужно
несколько специалистов - в некоторых случаях возникают спорные моменты
и требуется дополнительное мнение. Рассмотренные модели показали хорошие результаты производительности в своей области применения, с помощью
них можно эффективно обрабатывать, реконструировать медицинские данные, что отражает прогресс в разработке автоматических систем медицинской диагностики.



