\subsection*{MEDUSA: Multi-scale Encoder-Decoder Self-Attention Deep Neural Network Architecture for Medical Image Analysis}

% \subsection*{Ссылка} \url{https://arxiv.org/abs/2110.06063}
\subsection*{Введение}

Анализ медицинских изображений попрежнему сопряжен с трудностями, 
учитывая трудноразличимые грани между заболеваниями с похожими проявлениями, 
что увеличивает важность наличия качественной дифференциальной диагностики.
В данной работе исследуется концепция self-attention для повышения точности 
диагностики, особенно на ранних стадиях развития болезни.
\subsection*{Основная идея}
Авторы предполагают \cite{ann17}, что внедрение явного глобального контекста среди выборочного 
внимания на разных уровнях абстракций в нейросетевой архитектуре при способности различать 
локальный контекст на отдельных уровнях может привести к лучшей производительности.
Чтобы проверить предположение, предлагается MEDUSA (Multi-scale Encoder-Decoder Self-Attention) - 
self-attention механизм, приспособленный для задачи анализа медицинских изображений,
который явно использует связи между глобальным и масштабно-зависимым локальным контекстами 
с помощью реализации \glqq single body, multi-scale heads\grqq для повышения производительности.

\begin{minipage}{1.0\linewidth}
    \begin{center}
        \includegraphics[scale=0.5]{ann17_arch.png} \\
        \captionof{figure}{\scriptsize{
            Архитектура MEDUSA.
        }}
    \end{center}
    
\end{minipage} 

\subsection*{Данные}
CXR-2 Dataset
\subsection*{Результаты}

{\small

\begin{center}
    \captionof{table}{
            Sensitivity, PPV - positive predictive value и точность предложенного метода MEDUSA 
            в сравнении с другими сетями на данных из CXR-2 Dataset.
        }
    \begin{tabular}{|c|c|c|c|}
    \hline
         \textbf{Architecture} & \textbf{Sensitivity ($\%$)} & \textbf{PPV ($\%$)} & \textbf{Accuracy ($\%$)} \\
    \hline
    ResNet-50 & 88.50 & 92.20 & 90.50\\
    \hline
     COVID-Net& 93.50 & \textbf{100} & 94.00 \\
    \hline
     COVID-Net CXR-2 & 95.50 & 97.00 &  96.30 \\
    \hline
    SE-ResNet-50  & 90.50 & 98.90 & 94.75\\
    \hline
    CBAM  & 70.00 & \textbf{100} & 85.00\\
    \hline
    MEDUSA & \textbf{97.50} & 99.00 & \textbf{98.30}\\
    \hline
    \end{tabular}\par

\end{center}

}

\subsection*{Заключение}
Было показано, что предложенный метод показывает хорошие результаты как в предсказаниях так и в скорости в сравнении с 
уже существующими моделями.