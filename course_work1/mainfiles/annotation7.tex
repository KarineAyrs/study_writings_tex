\subsection*{Glioma Segmentation with Cascaded Unet}

% \subsection*{Ссылка} \url{https://arxiv.org/abs/1810.04008}
\subsubsection*{Введение}
Точная сегментация и реконструкция медицинских 3D изображений способны дать 
больше необходимой информации о прогрессировании заболевания и позволяют терапевту 
спланировать успешный курс лечения для больного. В данной работе \cite{Cascaded} авторы представляют
каскадный вариант популярной сети UNet \cite{Unet}, который итеративно улучшает результаты сегментации, 
полученные на предыдущих шагах. \par
\subsubsection*{Основная идея}
Предложенный метод может быть представлен как цепь классификаторов 
\(C_i\), одинаковой топологии F, у каждого из которых свой собственный 
набор параметров \(W_i\) для оптимизации в течение обучения.
Результат вычисления \(i\)-го шага представляется следующим образом: 
\(Y_i=F(X_i,Y_{i-1}, Y_{i-2}, W_i)\). Каждый из базовых блоков \(C_i\) - это
сеть архитектуры UNet, измененная для задачи сегментации глиом. В сравнении 
со стандартной архитектурой UNet, в предложенной модели используется несколько 
энкодеров, которые раздельно обрабатывают входные данные. Также, предложен метод объединения 
их выхода: в UNet \(i\)-й выход декодера зависит от выхода соответствующего 
энкодера и выхода предыдущего декодера - \(d_i^{t}=f(e_i^{t}, d^{t}_{i-1})\). 
Раскрывая первую свертку \(f\), получаем - \(d_i^{t}=g(W_{i,e}^{t}e_i^{t}+W_{i},d^{t}d^{t}_{i-1})\).
Далее предлагается объединить контекст, полученный на более низких слоях, 
добавляя соответствующий выход \(y^t\), поэтому \(d_i^{t}=g(W_{i,e}^{t}e_i^{t}+W_{i,d}^{t}d^{t}_{i-1}+W_{i,y}^{t}y^{t-i})\).
\\
\begin{minipage}{1.0\linewidth}
    \begin{center}
        \includegraphics[scale=0.6]{ann7_arch.png} \\
        % \caption{\scriptsize{Схематическое представление метода, описанного в статье.
        % T1, T2, T1ce, FLAIR - входные модальности МРТ-изображения, x4,x2 - понижающий
        % фактор входа сети. Пунткирные линии - соединения между блоками \(C_i\).}}
    \end{center}
\end{minipage}

\subsubsection*{Результаты}

Результат сегментации  оценивался по метрике Dice, отдельно вычисленной
для следующих частей опухоли: WT (whole tumor) -  вся опухоль, ET (enchancing tumor) - 
усиливающаяся часть опухоли и TC (tumor core) - ядро опухоли. \\
 \\
\begin{minipage}{1.0\linewidth}
    \begin{center}
        \includegraphics[scale=0.8]{ann7_res.png} \\
        % \caption{\scriptsize{Результаты без аугментации выходов}}
    \end{center}
\end{minipage}
 \\
\begin{minipage}{1.0\linewidth}
    \begin{center}
        \includegraphics[scale=0.6]{ann7_res1.png} \\
        % \caption{\scriptsize{Результаты с аугментацией выходов}}
    \end{center}
\end{minipage}
\subsubsection*{Заключение}

В данной работе был предложен алгоритм автоматической сегментации 
опухолей головного мозга по МРТ-зображениям, который решает также проблему
мультимодального входа и показывает хорошие резльтаты по сравнению с моделью UNet.


