\subsection*{Evidential segmentation of 3D PET/CT images}

% \subsection*{Ссылка} \url{https://arxiv.org/abs/2104.13293} 
\subsubsection*{Введение}
Из-за низкого разрешения и контрастности, результаты сегментации PET/CT изображений с помощью нейронных сетей не вызывают доверия. 
В данной работе авторы предлагают модель для сегментации диффузной B-крупноклеточной лимфомы из 
трехмерных PET/CT изображений, основанной на теории Демпстера-Шафера (BF) 
\textit{(Теория Демпстера — Шафера математическая теория очевидностей (свидетельств), 
основанная на функции доверия (belief functions) и функции правдоподобия (plausible reasoning), 
которые используются, чтобы скомбинировать отдельные части информации (свидетельства) для вычисления вероятности события)} и глубоком обучении. 
\subsubsection*{Основная идея}
\par
Архитектура предложенной нейронной сети (ES-Unet) \cite{ann4} основывается модуле UNet \cite{Unet}
для извлечения признаков (encoder-decoder) и модуле сегментации очевидностей 
(evidential segmentation - ES), которая основывается на одели evidential 
neural network и подходе, предложенных в ранних работах, для количественной 
оценки неопределенности относительно каждого вокселя решения с некоторой степенью 
доверия по функции массы Демпстера-Шаффера. Основная идея модуля ES - присвоить массу 
каждому из K классов и всему множеству классов \(\Omega\), основываясь на расстоянии между 
вектором признаков каждого вокселя и центрами прототипа \(I\). В процессе обучения сети 
минимизируется двусоставная функция потерь, позволяющая увеличить точность по мере Серенсена (Dice score) и уменьшить неопределенность.
\subsubsection*{Данные} 
\par
Датасет состоит из 173 изображений, полученных после исследования пациентов, у которых была диагностирована В-крупноклеточная лимфома.
\subsection*{Результаты} 
\par
Предложенная модель превосходит базовую модель UNet\cite{Unet}, так же как и другие модели (nnUnet, VNet, SegResNet). 
В частности, ES-Unet превосходит лучшую модель SegResNet на 1.9\%, 2.4\%, 1.4\% по Dice score, Sensitivity и F1 score соответственно.
\subsubsection*{Заключение}
Был разработан фреймворк ES-Unet для сегментации лимфом по трехмерным PET/CT изображениям с 
количественной оценкой неопределенности. Предложенная архитектура основывается на совмещении 
модели Unet и модуля ES. Обучение выполняется путем мнимизации двусоставной функции потерь. Р
азработанная модель справляется с поставленной задачей и превосходит по качеству предсказания 
уже существующие модели (Unet,nnUnet, VNet, SegResNet).