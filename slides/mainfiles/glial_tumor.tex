%!TEX root = ../slides.tex

\section{Глиальные опухоли}

\begin{frame}
  \frametitle{Глиальные опухоли}
  %\framesubtitle{Subtitles are optional.}
  % - A title should summarize the slide in an understandable fashion
  %   for anyone how does not follow everything on the slide itself.

  \begin{itemize}
  \item
  Глиальная опухоль является патологическим новообразованием, расположенным внутри мозга. 
  Она развивается из глии – вспомогательных клеток нервной ткани.
  
  \item
  Согласно классификации ВОЗ, глиальные опухоли делятся на четыре типа. В основе этой классификации лежит 4 основных признака:
    \begin{itemize}
      \item атипия клетки
      \item фигуры митозов
      \item наличие области некроза (отмирания тканей)
      \item разрастание эндотелия
    \end{itemize}
    
  \item Глиомы бывают:
    \begin{itemize}
      \item доброкачественными
      \item злокачественными
    \end{itemize}
     
  \end{itemize}
\end{frame}

\subsection{Методы обследования глиальных опухолей}

\begin{frame}
  \begin{itemize}
    \frametitle{Методы обследования глиальных опухолей}
    \item Компьютерная томография с контрастным усилением (КТ)
    \item Магнитно-резонансная томография  с контрастным усилением (МРТ)
    \item ПЭТ 
    \item Сцинтиграфия
    \item Неврологическое исследование, которое обязательно включает в себя офтальмологическую проверку остроты зрения, глазного дна и полей зрения
    \item Ангиография
  \end{itemize}
\end{frame}