%!TEX root = ../slides.tex
\section{Актуальность}
\begin{frame}
    \frametitle{Актуальность}
    \begin{itemize}
        \item Число глиом составляет более 60 \% от всех опухолей центральной нервной системы.
        мозга.
        \item Ранняя диагностика опухоли способствует высоким результатам лечения. Заболевание нуждается в дифференцировке от 
        гематомы внутри мозга, абсцесса, эпилепсии, прочих опухолевых процессов в центральной нервной 
        системе, последствий инсульта.
        \item Автоматическая сегментация/классификация глиальных опухолей головного мозга по ПЭТ-изображениям
        ускорит процесс дифференциальной диагностики и поможет определить вектор последующих исследований и лечения.
        
    \end{itemize}
\end{frame}

\section{Цель}
\begin{frame}
    \frametitle{Цель}
    \begin{itemize}
        \item Разработать метод искусственного интеллекта \(F\), который по 
        данным из обучающей выборки \(X=\{x_1,\dots,x_n\}\),
        состоящей из ПЭТ-изображений, возвращает результат предсказания \(y=F(X)\), 
        где y - это степень опасности опухоли.
    \end{itemize}
\end{frame}

\section{Постановка задачи}
\begin{frame}
    \frametitle{Постановка задачи}
    \begin{itemize}
        \item Сделать обзор существующих методов классификации и сегментации опухолей головного мозга,
        визуализирующихся на ПЭТ-изображениях.
        \item Найти и скомпилировать датасет, на которых будет производиться исследование.
        \item На полученном датасете применить методы сегментации/классификации из обзора, 
        сравнить результаты.
        \item Разработать собственный метод для улучшения процесса дифференциальной 
        диагностики глиальных опухолей.
        \item Провести эксперименты с полученным решением.
    \end{itemize}
\end{frame}